% Filming 101 document for TinkerMill.
% Daniel Williams <dwilliams@port8080.net>

% Command for compiling:
% pdflatex Filming101 && pdflatex Filming101

% Use latexclasses/tmarticle if this is being included from the git as a
% submodule.
\documentclass[letterpaper]{latexclasses/tmarticle}

% Call the style file here.

% Some document info.
\renewcommand{\titleinfo}{Filming 101}

% Here's the document.
\begin{document}
\maketitle

\newpage
\tableofcontents

\newpage

\section{Camera}

\subsection{Preparing the Camera}

\begin{itemize}
    \item
    Charge the battery.
    
    Make sure that the battery is fully charged before filming.  Plug the camera
    in the night before to make sure you have a full charge.  The worst thing
    that can happen during filming is having the camera die during a shot.
    Another handy thing to have is a spare battery.  Also bring the charger
    along just incase there is power to use.

    \item
    Format the memory.
    
    Make sure to format the card before shooting.  The second worst thing that
    can happen is to run out of film, or rather empty space on the flash card.
    The most reliable formating method is to use the one build into the camera.
    This will ensure that the filesystem on the flash card is compatible with
    the camera.  Make sure to carry spares also.

    \item
    Recording settings.
    
    % Put something here about settings.

\end{itemize}

\subsection{Zooming}

\begin{itemize}
    \item
    Don't zoom if possible.
    
    Camera motion is one of the hardest things to master.  If it doesn't have to
    happen, don't do it.
    
    \item
    Prepare the zoom.
    
    Figure out where the zoom will start and stop.  Make sure to practice the
    zoom a number of times.  Any inconsistencies will be easily noticed.
    
    \item
    Dolly Zoom.
    
    Also known as the Alfred Hitchcock zoom, in this technique the camera is
    moved while zooming the opposite direction in order to keep the subject of
    the shot the same size while distorting the surrounding environment.
    
    \item
    Ken Burns Effect.
    
    Used heavily in documentaries, this technique adds motion to otherwise
    static material.  This includes zooming and panning to keep the audience
    engaged.  This is usually used when showing still images on screen during a
    talkover.

\end{itemize}

\subsection{Whitebalance}

\begin{itemize}
    \item
    Automatic.
    
    \item
    Basic Settings.
    
    \item
    Custom Adjustment.

\end{itemize}

\subsection{Basic Staging}

\begin{itemize}
    \item
    Wide shots.
    
    \item
    Interviews.
    
    \item
    Checking Lighting.

\end{itemize}

\section{Editing}

\subsection{Importing onto the Computer}

\begin{itemize}
    \item
    Copy the files.
    
    \item
    Open the project.
    
    \item
    Import the files.

\end{itemize}

\subsection{Basic Editing}

\begin{itemize}
    \item
    The interface.
    
    \item
    Non-linear editing.
    
    \item
    Transitions.

\end{itemize}

\subsection{Sharing your Content}

\begin{itemize}
    \item
    Exporting.
    
    \item
    Uploading.

\end{itemize}

\end{document}
